\section{Keyword-Driven}

Keyword-driven testing is a logical extension to data-driven testing~\cite{Fewster99}.
Besides test data, it also makes use of keywords in order to instruct how the
data is read from the external data source and to be interpreted by the test
scripts. It takes the concept of data-driven testing even further~\cite{Lau07}
by adding keywords driving the test executing into the test data with the desire
to be able to specify automated test cases without having to specify all the
excessive detail. The test data file is expanded and it becomes a description of
the test case using a set of keywords to indicate the tasks to be performed.
Obviously, the driver script has to be able to interpret the keywords in order
to execute the desired test case.

\begin{table}[!ht]
\centering
\begin{tabular}{lll}
\textbf{Test Case} & \textbf{Keyword} & \textbf{Value} \\
Addition 01 & & \\
& set\_operand & 5 \\
& set\_operator & + \\
& set\_operand & 5 \\
& check\_result & 10 \\
Addition 02  & & \\
& set\_operand & 5 \\
& set\_operator & + \\
& set\_operand & 10 \\
& set\_operator & - \\
& set\_operand & 1 \\
& check\_result & 14 \\
\end{tabular}
\caption{Keyword-driven test data with mathematical operations.}
\label{table:tab2}
\end{table}

Back to the previous scenario on data-driven testing where the operation
10 * 5 - 5 = 45 would require major changes to both test data and driver
scripts. The keyword-driven testing is the perfect technique to perform that
kind of scenarios. They could be extended with no further development since the
keywords are already implemented by the driver script. This way, the driver
script is no longer tied to a particular feature of the software under test,
nor indeed to a particular application or system~\cite{Fewster99}.

This technique takes a descriptive approach to test case implementation since
the test file describes the test case, by stating what the test case does and
not how it does it. To implement an automated test case we have only to provide
a description of the test case more or less as we would do for a knowledgeable
human tester. By using this approach we can build knowledge of the system under
test into our test automation environment. People with business knowledge can
concentrate on the test files while people with technical skills can concentrate
on the driver scripts since it is possible to develop the test cases separately
from the test scripts due to the abstraction created by the test files.
