\section{Model-Driven Testing}

Model-based testing is the automation of black-box test design. 
A model-based testing tool uses various test generation 
algorithms and strategies to generate tests from a 
behavioral model of the system under test (SUT)~\cite{1200168}.

With model-based testing we should be able to easily generate 
a larger test suite from the same model or regenerate the test
suite each time the system requirements change.

This approach to software
development is increasingly gaining the attention of
both industry and academia. Actualy, is considered as 
leading-edge technology in industry. Unlike traditional
development techniques which tend to focus on
implementation, model-driven software development
stresses the use of models at all levels of the software
development process~\cite{5381477}.

\subsection{Types}
There exist many model-driven testing approaches
and tools, which vary significantly in their specific designs,
testing target, tool support, and evaluation strategies.
 
\subsection{Modelling}
\subsection{Challenges in Modelling}
Model-driven testing can improve considerably the test 
efficiency and test quality when the models are light 
and available. The elaboration of a model is, probably, 
the key for the success of model-driven testing in practice,
and if done wrongly can lead to a set of undesirable 
outcomes~\cite{Peleska.2013}:

\begin{enumerate}
\item If complex models have to be completed before testing
can start, this induces an unacceptable delay for the
proper test executions.
\item For complex SUT, like systems of systems, test models 
need to abstract from a large amount of detail, because 
otherwise the resulting test model would become unmanageable.
\item The required skills for test engineers writing test 
models are significantly higher than for test engineers 
writing sequential test procedures.
\end{enumerate}

\subsection{Model-Based Testing and Agile methods}
In the eyes of the agile practitioners, writing acceptance tests
that specify what the system is supposed to do, brings lot of value:
\begin{itemize}
\item It forces the customer to specify precisely what is required;
\item When the acceptance tests are “green” the customer has much 
more confidence that real and useful work has been done;
\item An executable specification gives a clear and measurable 
objective to the development team.
\end{itemize}

Having the acceptance tests centered around a model it makes 
not only easier to develop a significant number of acceptance tests,
but also to change the model and regenerate those same tests.

We can benefit even more from mixing model-based and acceptance tests, if the model itself is built around some agile principles:
\begin{itemize}
\item Test models should have a precise purpose;
\item Test models should be light; 
\item Test models should grow incrementally through an iterative approach;
\item Test models encourage discussion about the exact behavior of the system;
\item Test models should not be used for documentation.
\end{itemize}

\subsection{Benefits}
Model-based testing can lead to less time and effort spent
on testing if the time needed to write and maintain the model 
plus the time spent on directing the test generation is less 
than the cost of manually designing and maintaining a test suite.

\subsection{Known Limitations}
Model-based testing is not as trivial as other testing techniques,
and requires a steep learning curve and different testing 
skills: modeling and programming skills. It is desirable to have 
a reasonably mature testing process and some experience with 
automated test execution before fully adopt this approach.

Up to date requirements plays a big role in model-based testing.
The test cases are very coupled to the requirements, and
requirements change, often. If the model is build based on outdated
requirements, it will lead to unreliable results. 

When one of these tests
fails, it's hard to check if the failure is caused by the SUT, the
adaptor, or even from the model itself. This process makes more
difficult and time-consuming to find the root cause of the failed test.

Since this approach can generate huge numbers of tests, it becomes
necessary to move toward other measurements of test progress instead
of relying on the number-of-tests metric. Business code, requirements and model
coverage are some of the alternatives.

