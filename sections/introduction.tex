\section{Introduction}

Software must be tested to have confidence that it will work as it should in
its intended environment. Software testing needs to be effective at finding
any defects which are there, but it should also be efficient, performing the
tests as quickly and cheaply as possible. Software testing can provide an
objective, independent view of the software quality and performance that allows
the business stakeholders to understand the quality of software implementation.
It has been considered as a time consuming activity that calls for enhanced and
powerful test techniques with the intent of finding software bugs. However, automating
software testing can significantly reduce the effort required for adequate
testing, and significantly reduce the time needed to run tests that would take
hours to run manually.

Automating tests is a skill, but a very different skill from testing. Many
organizations find that it is more expensive to automate a test than to perform
it once manually. In order to gain benefits from test automation, the tests to
be automated need to be carefully selected and implemented. Automating a test
affects how economic and evolvable it is. Once implemented, the cost of running
an automated test is much more cheaper than perform it manually. However,
automated tests generally cost more to create and maintain. The better the
approach to automating tests, the cheaper it will be to implement them in the
long term.

There are a number of ways in which testing tools can automate parts of test
case design. These tools are sometimes called test input generation tools and
their approach is useful in some contexts, but they will never completely
replace the intellectual testing activities. One problem with all test case
design approaches is that the tool may generate a very large number of tests.
Some tools include ways of minimizing the tests generated against the specified
criteria. However the tool may still generate far too many tests to be run in a
reasonable time. Furthermore, it cannot distinguish which tests are the most important,
since it requires creative intelligence only available from humans. Test
generation tools rely on algorithms to generate the tests. The tools will be
more thorough and more accurate than a human tester using the same algorithm,
so this is an advantage. However, a human being will think of additional tests
to try, may identify aspects or requirements that are missing, may be able to
identify where the specification is incorrect based on personal knowledge and
may be able to determine which are the most important test cases to run. The
best use of test generation tool is when the scope of what can and cannot be
done by them is fully understood.

In this paper, we will explore and compare some of the techniques and strategies
used to take full advantage of testing automation.