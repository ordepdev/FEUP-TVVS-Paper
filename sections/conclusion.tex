\section{Conclusion}

With all the excitement about test generation, it is important to remember that
all of the discussed techniques are still about black-box functional testing and
that they do not replace the need for testers. On the contrary, it gives testers
new weapons to explore the space of infinite possibilities to test a system.

There must be a commitment to support test automation. In order to make test
automation work, and to keep it working, both testers and management have to be
willing and able to make changes in the established development process. Testers
must be willing to learn new methods, measure their work, and
commit to creating and maintaining high quality, reusable test scripts.
Management must be willing to make time for tester education and to
recognize the importance and value of their testing resources.

Test automation is not the same as testing, and the skills needed to be a
good test automation engineer are not the same skills needed to be a good tester.
Testing should fit into the software development process at every stage.
Test cases should be identified and designed on the left-hand side of the
V-model, but can only be executed and compared after the software is
available for testing.

There are tools for each approach that we discussed in this paper and for all
types of test activities throughout the development life cycle, though none of
them can make any testing activity completely automatic.

The test automation holds great promise and can give significant
benefits, such as repeatable consistent regression tests that are run more
often, testing of attributes that are difficult to test manually, better use of
resources, reuse of tests, earlier time to market, and increased confidence.