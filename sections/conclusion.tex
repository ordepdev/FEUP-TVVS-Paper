3 Lessons learned: organizational issues
13.6.3.1 There must be a commitment to support test automation
In order to make test automation work, and to keep it working, both testers
and management have to be willing and able to make changes in the established
development process.
Testers must be willing to learn new methods, measure their work, and
commit to creating and maintaining high quality, reusable test scripts.
Management must be willing to make time for tester education and to
recognize the importance and value of their testing resources. 

Finding the time to automate testing
The reality of commercial software development is that testers must be
ready to test when they receive the code from the developers.
Virtually all the current software development models advocate the
involvement of the testers during the design and development phases of the
project. Prodigy has always encouraged this early tester involvement as
well. But there had been no satisfactory method for communicating design
and implementation details that was both clear and reliable.
The problem is that maintaining formal detailed documentation, design
specifications, and test plans is too cumbersome and time consuming for
use in an environment where development can complete and discard six or
seven prototypes in a day and testers may be expected to test seven versions
in a week.
Since there was no mechanism for conducting and documenting a
meaningful developer-tester dialog, the results of early tester involvement
were often disappointing. Testers were not normally able to gain a sufficiently
detailed description of the software function to size the effort, let
alone begin detailed planning. 

13.6.3.4 Reuse testing resources
Test scripts and test plans cannot become 'disposable'. They are as durable
as source code and design documentation. What would be the reaction to a
developer who insisted on rewriting every program that had an error or
every program that came from some other developer?
Ideally, test scripts and key traces should accompany code through the
entire project life cycle. Each group in the project life cycle should be able
to add to or refine the repository of available test scripts.
The test base benefits greatly from the diverse perspectives and priorities
of the contributing testers. In this way the test repository grows along
with the system, becoming the best defense against the ever-growing complexity
of the software. 

Supplying testers with tools alone does not insure that test automation will
take place. Several things are necessary to sustain test automation:
1. methods are more important than tools;
2. there are many testing tasks that must be automated besides test script
capture and replay;
3. the test tools must support a structured test methodology and automate
the measuring of software;
4. greater value must be placed on test resources. If test scripts are not
maintained for reuse there is little value in script replay automation;
5. management has a key role to play in enabling automation. They must
insure that testers receive the education that they need to succeed and
that the development cycle is adjusted to accommodate the pre-testing
tasks required by automated test efforts. Management must also expect
software measurements and cumulative results reports from testers.
The benefits of good testing practice together with test automation are
significant:
1. test efforts can be reduced dramatically, from 30% to 50%;
2. the quality of software testing can be greatly improved through the use
of structured test methods and software metrics;
3. through the use of measurements, project sizing and scheduling esti
mates can be greatly improved.
Test automation is a part of the product development life cycle, a complex
dynamic process spanning years. The end is not in sight. Continuing support
and commitment must be provided if software test automation efforts
are to succeed. The most important thing that we learned was that testing is
much more important than automation. Without good testing methods and
metrics, test automation may not succeed. 

Test automation is not the same as testing, and the skills needed to be a
good test automator are not the same skills needed to be a good tester.
Testing should fit into the software development process at every stage.
Test cases are best identified and designed early (on the left-hand side of the
V-model), but can only be executed and compared after the software is
available for testing (the right-hand side of the V-model).
Tool support is available for all types of test activities throughout the
development life cycle, though none of them can (or will ever be able to)
make any testing activity completely automatic.
The automation of testing holds great promise and can give significant
benefits, such as repeatable consistent regression tests that are run more
often, testing of attributes that are difficult to test manually, better use of
resources, reuse of tests, earlier time to market, and increased confidence.
However, many problems typically arise when attempting to automate testing,
including unrealistic expectations, poor testing practices, a false sense of
security, maintenance costs, and other technical and organizational problems.
Although there are ways to support test case design activities, we consider
that automating the more clerical and tedious aspects of testing
generally gives more benefit than trying to automate the more intellectual
test activities. This book therefore concentrates on the test activities of executing
test cases and comparing outcomes. This is what we mean by
automating software testing.
Test automation has its limitations. It does not replace manual testing, and
manual testing will find more defects than automated testing. There is a
greater reliance on the correctness of expected outcomes. Test automation
does not improve test effectiveness, and can limit software development
options. The testing tools have no imagination and are not very flexible.
However, test automation can significantly increase the quality and productivity
of software testing